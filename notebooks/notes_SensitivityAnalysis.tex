%\documentclass{article}
\documentclass[prb,superscriptaddress,amsmath]{revtex4}
\usepackage{graphicx}
\usepackage{amssymb}
\usepackage{color}

\bibliographystyle{unsrt}
\begin{document}

\title{Automating Sensitivity Analysis in the Delphi pipeline}

\maketitle

\section{Introduction}

There exist a growing need to  quantitatively understand  the scope and behavior of
physical models that aim to study complex phenomena. However, an over-abundance of such
models can often present challenges as they span a wide range of parameter
spaces. Further, the complexity of such mechanistic models, both dynamic
and static, make the structural relationship between the parameters and model output
non-trivial thereby preventing one from identifying and estimating the effects
of the most relevant parameters. Senstitivity analysis provides a route for
overcoming this challenge by determining the effects of parameter uncertainty
on models in a non-intrusive manner which extends its applicability to any
general system.

The aim of the AutoMATES initiative is to integrate different
computational models under a common semantically rich representation from  a
variety of sources (text, equations, and source code) that will
allow domain experts to perform interesting analysis on scientific models \cite{Adarsh20a}. This
typically involves four different components viz., Extraction, Grounding,
Comparison, and Augmentation.
Extraction and Grounding together are responsible for creating a generalized intermediate
representation (Grounded Function Network - GrFN) of source codes that are linked to information about the
underlying models, equations, as well as the
physical variables extracted from scientific documents and other relevant text. Once such a unified framework has been
created, independent models can be compared (Comparison) and even augmented (Augmentation) for validation purposes.
Through our GrFN representation,
we can analyze the structural and functional properties of the network that will allow us to
identify regions of parameter spaces where
our selected models display similar features/properties.


\section{Sensitivity Analysis}

The robustness of any model that is dependent on a set of input variables can be tested by performing a sensitivity analysis.
Sensitivity analysis is a measurement of the influence of the input parameters
on the variability in model output. It therefore not only allows the experimenter to
understand the structural relationship between the input and output but also
acts a filtering mechanism to identify the most dominant input parameters 
in model uncertaintiy quantification.
In general, there are two types of sensitivity analysis - i) local, ii) global.
Local sensitivity analysis refers to the measurement of model output
variability as a result of perturbations within a single input parameter.
Global sensitivity analysis involves the estimation of changes in the response
function due to variations in the entire input parameter space.
In this article, we shall restrict ourselves to the discussion of global
sensitivity methods. Again, there are multiple methods that perform  global
sensitivity analysis \cite{Saltelli07a}, namely, weighted average local sensitivity analysis,  partial
rank correlation coefficient, multiparameteric sensitivity  analysis, Fourier
amplitude sensitivity analysis (FAST) \cite{Saltelli99a, Chan97a} and lastly, Sobol's method
\cite{Sobol01a, Saltelli02a, Saltelli10a}. 
Sobol's method is based on ANOVA (Analysis of Variances) represenation wherein
the model output is decomposed into summands of
variance in the  input parameters with different dimensions \cite{Sobol90a}. This global method 
is particularly useful as it can determine the relative contributins of
individual parameters as well as the interaction between them, as given by
the integrands of higher dimensions, to the overall output variance.
The decomposition of the generalized model output $f(x)$ is shown below.

\begin{equation}
f(x) = f_0 + \sum_i f_i(x_i) + \sum_{i < j} f_{ij}(x_i, x_j) + ... + f_{12
...n} (x_1, x_2, ..., x_n)
\end{equation}

Here, $x_1, x_2, ... x_n$ are the $n$ independent model input parameters that
span a unit-dimensional hypercube. 
Under the assumption that $f(x)$ is square integrable, in addition to the constraint
below (ANOVA representation), 
\begin{equation}
\int_{0}^{1} f_{i_1 ...i_s} (x_{i_1},...x_{i_s}) dx_k = 0
\label{ANOVA}
\end{equation}
 where $k = i_1, ... i_s$, one computes the variances of $f(x)$ ($D$) and $f_{i_1, ... , i_s} (x_{i_1}, ...,
x_{i_s})$ ($D_{i_1 ... i_s}$) to be the following:
\newline
\begin{equation}
D = \int f^2 dx - f_o^2
\end{equation}
and
\begin{equation}
D_{i_1, ..., i_s} = \int f^2_{i_1, ..., i_s} dx_{i_1}, ... dx_{i_s}
\end{equation}

The global sensitivity indices are now given by $S_{i_1}...{i_s}$ in the equation
below:

\begin{equation}
S_{i_1}...{i_s} = \frac {D_{i_1}...{i_s}}{D}
\end{equation}

The integer ``s'' is the dimension or the order of the sensitivity index.
Since, the sensitivity indices are essentially a measure of the output
variance as a function of the input features, they must be non-negative.
Under the assumption of independent input variables, the sum of all sensitivity
indices should be one \cite{Sobol01a, Glen12a}.

The first order Sobol index which measures the direct influence of a single
parameter on the output variance is therefore given by $S_i = \frac{D_i}{D}$
or alternatively, $S_i = \frac{\mathbb{V}(\mathbb{E}(Y|X_i))}{\mathbb{V}(Y)}$.
$Y$ ($X_i$) is the output (input), $\mathbb{E}(Y|X_i)$ measures the conditional
expectation of $Y$ with $X_i$ fixed whereas $\mathbb{V}(Q)$ is the variance of
$Q$ \cite{Saltelli10a}.
While one can in principle calculate the indices upto all possible orders, for
practical purposes, computation of first ($S_1$) and second ($S_2$) order as
well as the total ($S_T$)
Sobol indices is considered to be standard. 

Computation of Sobol indices is done through a quasi-Monte Carlo method. The
first step involves the generation of a quasi-random, low discrepancy sequence
of numbers that are sampled uniformly which lead to faster convergence and
higher accuracy  than a
naive Monte Carlo approach. These sequences are known as Sobol sequences \cite{Sobol67a}.
Following this, the model output is evaluated from the sampled data points.
This process is repeated thousands of times to yield the expectation and
variance from the sets of model outputs. The number of samples required to
achieve a high  degree of accuracy depend on the number of input parameters.   

\bibliography{sensitivity}

\end{document}
