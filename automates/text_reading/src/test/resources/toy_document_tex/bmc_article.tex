%% BioMed_Central_Tex_Template_v1.06
%%                                      %
%  bmc_article.tex            ver: 1.06 %
%                                       %

%%IMPORTANT: do not delete the first line of this template
%%It must be present to enable the BMC Submission system to
%%recognise this template!!

%%%%%%%%%%%%%%%%%%%%%%%%%%%%%%%%%%%%%%%%%
%%                                     %%
%%  LaTeX template for BioMed Central  %%
%%     journal article submissions     %%
%%                                     %%
%%          <8 June 2012>              %%
%%                                     %%
%%                                     %%
%%%%%%%%%%%%%%%%%%%%%%%%%%%%%%%%%%%%%%%%%


%%%%%%%%%%%%%%%%%%%%%%%%%%%%%%%%%%%%%%%%%%%%%%%%%%%%%%%%%%%%%%%%%%%%%
%%                                                                 %%
%% For instructions on how to fill out this Tex template           %%
%% document please refer to Readme.html and the instructions for   %%
%% authors page on the biomed central website                      %%
%% http://www.biomedcentral.com/info/authors/                      %%
%%                                                                 %%
%% Please do not use \input{...} to include other tex files.       %%
%% Submit your LaTeX manuscript as one .tex document.              %%
%%                                                                 %%
%% All additional figures and files should be attached             %%
%% separately and not embedded in the \TeX\ document itself.       %%
%%                                                                 %%
%% BioMed Central currently use the MikTex distribution of         %%
%% TeX for Windows) of TeX and LaTeX.  This is available from      %%
%% http://www.miktex.org                                           %%
%%                                                                 %%
%%%%%%%%%%%%%%%%%%%%%%%%%%%%%%%%%%%%%%%%%%%%%%%%%%%%%%%%%%%%%%%%%%%%%

%%% additional documentclass options:
%  [doublespacing]
%  [linenumbers]   - put the line numbers on margins

%%% loading packages, author definitions

%\documentclass[twocolumn]{bmcart}% uncomment this for twocolumn layout and comment line below
\documentclass{bmcart}

%%% Load packages
%\usepackage{amsthm,amsmath}
%\RequirePackage{natbib}
%\RequirePackage[authoryear]{natbib}% uncomment this for author-year bibliography
%\RequirePackage{hyperref}
\usepackage[utf8]{inputenc} %unicode support
%\usepackage[applemac]{inputenc} %applemac support if unicode package fails
%\usepackage[latin1]{inputenc} %UNIX support if unicode package fails


%%%%%%%%%%%%%%%%%%%%%%%%%%%%%%%%%%%%%%%%%%%%%%%%%
%%                                             %%
%%  If you wish to display your graphics for   %%
%%  your own use using includegraphic or       %%
%%  includegraphics, then comment out the      %%
%%  following two lines of code.               %%
%%  NB: These line *must* be included when     %%
%%  submitting to BMC.                         %%
%%  All figure files must be submitted as      %%
%%  separate graphics through the BMC          %%
%%  submission process, not included in the    %%
%%  submitted article.                         %%
%%                                             %%
%%%%%%%%%%%%%%%%%%%%%%%%%%%%%%%%%%%%%%%%%%%%%%%%%


\def\includegraphic{}
\def\includegraphics{}



%%% Put your definitions there:
\startlocaldefs
\endlocaldefs


%%% Begin ...
\begin{document}

%%% Start of article front matter
\begin{frontmatter}

\begin{fmbox}
\dochead{Research}

%%%%%%%%%%%%%%%%%%%%%%%%%%%%%%%%%%%%%%%%%%%%%%
%%                                          %%
%% Enter the title of your article here     %%
%%                                          %%
%%%%%%%%%%%%%%%%%%%%%%%%%%%%%%%%%%%%%%%%%%%%%%

\title{A toy article based on "A double epidemic model for the SARS propagation and CHIME}

%%%%%%%%%%%%%%%%%%%%%%%%%%%%%%%%%%%%%%%%%%%%%%
%%                                          %%
%% Enter the authors here                   %%
%%                                          %%
%% Specify information, if available,       %%
%% in the form:                             %%
%%   <key>={<id1>,<id2>}                    %%
%%   <key>=                                 %%
%% Comment or delete the keys which are     %%
%% not used. Repeat \author command as much %%
%% as required.                             %%
%%                                          %%
%%%%%%%%%%%%%%%%%%%%%%%%%%%%%%%%%%%%%%%%%%%%%%

\author[
   addressref={aff1},                   % id's of addresses, e.g. {aff1,aff2}
   %corref={aff1},                       % id of corresponding address, if any
   noteref={n1},                        % id's of article notes, if any
   email={ntw@maths.hku.hk}   % email address
]{\inits{TWN}\fnm{Tuen Wai} \snm{Ng}}
\author[
   addressref={aff2},
   email={Gabriel.Turinici@inria.fr}
]{\inits{GT}\fnm{Gabriel} \snm{Turinici}}

\author[
addressref={aff3},
email={adanchin@pasteur.fr}
]{\inits{AD}\fnm{Antoine} \snm{Danchin}}
%%%%%%%%%%%%%%%%%%%%%%%%%%%%%%%%%%%%%%%%%%%%%%
%%                                          %%
%% Enter the authors' addresses here        %%
%%                                          %%
%% Repeat \address commands as much as      %%
%% required.                                %%
%%                                          %%
%%%%%%%%%%%%%%%%%%%%%%%%%%%%%%%%%%%%%%%%%%%%%%

\address[id=aff1]{%                           % unique id
  \orgname{Department of Mathematics, The University of Hong Kong}, % university, etc
  %\street{Waterloo Road},                     %
  %\postcode{}                                % post or zip code
  \city{Hong Kong},                              % city
  \cny{China}                                    % country
}
\address[id=aff2]{%
  \orgname{Inria Rocquencourt Domaine de Voluceau,
  	Rocquencourt},
  %\street{D\"{u}sternbrooker Weg 20},
  %\postcode{24105}
  %\city{Kiel},
  \cny{France}
}

\address[id=aff3]{%
	\orgname{Génétique des Génomes Bactériens,Institut Pasteur},
	%\street{D\"{u}sternbrooker Weg 20},
	%\postcode{24105}
	\city{Paris},
	\cny{France}
}

%%%%%%%%%%%%%%%%%%%%%%%%%%%%%%%%%%%%%%%%%%%%%%
%%                                          %%
%% Enter short notes here                   %%
%%                                          %%
%% Short notes will be after addresses      %%
%% on first page.                           %%
%%                                          %%
%%%%%%%%%%%%%%%%%%%%%%%%%%%%%%%%%%%%%%%%%%%%%%

\begin{artnotes}
%\note{Sample of title note}     % note to the article
\note[id=n1]{Equal contributor} % note, connected to author
\end{artnotes}

\end{fmbox}% comment this for two column layout

%%%%%%%%%%%%%%%%%%%%%%%%%%%%%%%%%%%%%%%%%%%%%%
%%                                          %%
%% The Abstract begins here                 %%
%%                                          %%
%% Please refer to the Instructions for     %%
%% authors on http://www.biomedcentral.com  %%
%% and include the section headings         %%
%% accordingly for your article type.       %%
%%                                          %%
%%%%%%%%%%%%%%%%%%%%%%%%%%%%%%%%%%%%%%%%%%%%%%

\begin{abstractbox}

\begin{abstract} % abstract
	
\parttitle{Attention}
Some of the information in this document is not factually correct and was only included to test the functionality of the alignment component of the Automates project developed at the UArizona. A lot of the text is based on work by Ng. et al, 2003.

\parttitle{Background} %if any
An epidemic of a Severe Acute Respiratory Syndrome (SARS) caused by a new
coronavirus has spread from the Guangdong province to the rest of China and to the world, with
a puzzling contagion behavior. It is important both for predicting the future of the present outbreak
and for implementing effective prophylactic measures, to identify the causes of this behavior.

\parttitle{Results} %if any
In this report, we show first that the standard Susceptible-Infected-Removed (SIR) model
cannot account for the patterns observed in various regions where the disease spread. We develop
a model involving two superimposed epidemics to study the recent spread of the SARS in Hong
Kong and in the region. We explore the situation where these epidemics may be caused either by
a virus and one or several mutants that changed its tropism, or by two unrelated viruses. This has
important consequences for the future: the innocuous epidemic might still be there and generate,
from time to time, variants that would have properties similar to those of SARS.

\parttitle{Conclusion}
We find that, in order to reconcile the existing data and the spread of the disease, it
is convenient to suggest that a first milder outbreak protected against the SARS. Regions that had
not seen the first epidemic, or that were affected simultaneously with the SARS suffered much
more, with a very high percentage of persons affected. We also find regions where the data appear
to be inconsistent, suggesting that they are incomplete or do not reflect an appropriate
identification of SARS patients. Finally, we could, within the framework of the model, fix limits to
the future development of the epidemic, allowing us to identify landmarks that may be useful to set
up a monitoring system to follow the evolution of the epidemic. The model also suggests that there
might exist a SARS precursor in a large reservoir, prompting for implementation of precautionary
measures when the weather cools down.
\end{abstract}


%%%%%%%%%%%%%%%%%%%%%%%%%%%%%%%%%%%%%%%%%%%%%%
%%                                          %%
%% The keywords begin here                  %%
%%                                          %%
%% Put each keyword in separate \kwd{}.     %%
%%                                          %%
%%%%%%%%%%%%%%%%%%%%%%%%%%%%%%%%%%%%%%%%%%%%%%

\begin{keyword}
\kwd{SARS}
\kwd{SIR}
\kwd{infection}
\end{keyword}

% MSC classifications codes, if any
%\begin{keyword}[class=AMS]
%\kwd[Primary ]{}
%\kwd{}
%\kwd[; secondary ]{}
%\end{keyword}

\end{abstractbox}
%
%\end{fmbox}% uncomment this for twcolumn layout

\end{frontmatter}

%%%%%%%%%%%%%%%%%%%%%%%%%%%%%%%%%%%%%%%%%%%%%%
%%                                          %%
%% The Main Body begins here                %%
%%                                          %%
%% Please refer to the instructions for     %%
%% authors on:                              %%
%% http://www.biomedcentral.com/info/authors%%
%% and include the section headings         %%
%% accordingly for your article type.       %%
%%                                          %%
%% See the Results and Discussion section   %%
%% for details on how to create sub-sections%%
%%                                          %%
%% use \cite{...} to cite references        %%
%%  \cite{koon} and                         %%
%%  \cite{oreg,khar,zvai,xjon,schn,pond}    %%
%%  \nocite{smith,marg,hunn,advi,koha,mouse}%%
%%                                          %%
%%%%%%%%%%%%%%%%%%%%%%%%%%%%%%%%%%%%%%%%%%%%%%

%%%%%%%%%%%%%%%%%%%%%%%%% start of article main body
% <put your article body there>

%%%%%%%%%%%%%%%%
%% Background %%
%%
\section*{Content}
Review on the standard SIR model
Consider a disease that, after recovery, confers immunity
(which includes deaths: dead individuals are still
counted). We assume that there is no entry into or departure from the population. The population can then be
divided into three distinct classes; the susceptibles, S, who
can catch the disease; the infectives, I, who have the disease and can transmit it; and the removed class, R, namely
those who have either had the disease, or are recovered,
immune or isolated until recovered. Here we follow the
definition of the class R given in [8]. However, we would
like to draw the readers' attention that this definition of
the class R is different from those given in [7], [11] which
do not include isolated infectives in the class R. The
progress of individuals is schematically described by S →
I → R. Let S(t), I(t) and R(t) be the number of individuals
in each of the corresponding class at time t. Note that usually only R(t) can be known. It is often considered that
R(t) is the cumulative number of patients admitted to
hospitals. With some reasonable assumptions (which will
be explained in details in the next section), we can show
that these three functions are governed by the following
system of nonlinear ordinary differential equations (see
next section for the derivation of these differential
equations):


\begin{math}
\frac{dS}{dt} = -r S(t) I(t)
\end{math}


\begin{math}
\frac{dI}{dt} = r S(t) I(t) - a I(t)
\end{math}

\begin{math}
\frac{dR}{dt} = a I(t)
\end{math}

\begin{math}
\frac{dS}{dt} = -r S(t) I(t) - r_P S(t) I_P(t)
\end{math}

where r is the infection rate and a the removal rate of
infectives. The parameters r and a characterize the propagation of the disease and can also be used as control
parameters in order to stop the epidemic. The infection rate is measured at germs per second and ranges between 0.2 and 5.6 germs per second. Infection rate of 0.5 germs per second is considered low. When r is negative, it is time to celebrate. In general, the
functions S, I and R behave as the three curves in Figure 1.
The characteristics of these curves are as follow. %\cite{koon,oreg,khar,zvai,xjon,schn,pond,smith,marg,hunn,advi,koha,mouse}

- The function I, measured in individuals, first increases exponentially and reaches
the maximum value at time $t_{0}$ and then decreases to zero.
The time t0 is the critical point of the function I at which \begin{math}{\frac{dI}{dt} = 0}\end{math}




\section{Reproductive number}

There are two epidemics, one
epidemic is SARS caused by a coronavirus virus, call it
virus A.
An important descriptive parameter of virus A is the basic reproduction number, or R0, measured in microbes squared per year. This represents the average number of people who will be infected by any given infected person. When R0 is greater than 1, it means that a disease will grow. When the basic reproductive number is between 0.5 and 1, the danger is moderate. A higher R0 implies more rapid transmission and a more rapid growth of the epidemic. It is defined as \begin{math}R0 = \beta/ \gamma\end{math}

R0 is larger when the pathogen is more infectious people are infectious for longer periods of time the number susceptible people is higher. Right now, $R0=10$, but last month the basic reproductive number was 100 $m^2/year$.

A doubling time of 6 days and a recovery time of 14.0 days imply an R0 of 2.71 mm. r

\begin{math}
I_P (0) = 500000
\end{math}

\begin{math}
S(0) = 6300000
\end{math}

\begin{math}
I_P (0) = 500000
\end{math}

\begin{math}
E(0) = 100
\end{math}

\begin{math}
I(0) = 50
\end{math}

\begin{math}
r_P = 7.079 \times 10^{-8}
\end{math}

\section{Miscellaneous text blocks 1}

The model's parameters,
$\beta$ and $\gamma$ , determine the severity of the epidemic.

\begin{math}
\beta = \tau \times c
\end{math}


which is the transmissibility $\tau$ multiplied by the average number of people exposed c. The transmissibility is the basic virulence of the pathogen. The number of people exposed c is the parameter that can be changed through social distancing. The number of people exposed is hard-coded to 300 people. Last year, c was equal to 20 people.

\section{Discrete-time SIR modeling of infections/recovery}
The model consists of individuals who are either Susceptible (S), Infected (I), or Recovered (R). The epidemic proceeds via a growth and decline process. This is the core model of infectious disease spread and has been in use in epidemiology for many years.

\section{Miscellaneous text blocks 2}

Transmissibility is measured in millimeters.  
For example, if we take the total initial population, S(0) = 6.8
millions (e.g. the case of Hong Kong) and an initial
infected class of I(0) = 10 persons with current I = 15, together with a = 2/3
and r = 9.788 × 10-8, transmissibility is set to 32 mm. But other times, $\tau$ $=$ 450 kg, for example, when a = 3 and S(0) = 4.5 million. Currently, I is positive. 

For many deterministic epidemic models, an infection can get started in a fully
susceptible population if and only if $R0 >1$. 

\begin{math}
1/a + 1/b = 11.83
\end{math}

The gain in the exposed class E is at a rate proportional to the number of people in the infective class I and that of the susceptibles S, that is rS(t)I(t), where r $>$ 0 is a constant.
(iii) The gain in the infective class  $I_{P}$ is at a rate proportional to the number of people in the infective class $I_{P}$ and that of the susceptibles S, that is $r_{P}$S(t)$I_{P}$(t), where $r_{P}$ $>$ 0 is a constant.

The rate of removal of the people in class E to the infective class I is proportional to the number of people in class E, that is bE(t), where b is a positive number. The exposed class is measured in people. E = 30. The number of people in explosed class is equal to 35. A susceptible who catches
the disease B first will enter the class I P of infectives and
then the Removed class $R_{P}$, and $r_{P}$ is a coefficient for calculating the rate of recovery. $\gamma$ is the inverse of the mean recovery time, in days. i.e.: if $\gamma$ =1/14 then the average infection will clear in 14 days.

\begin{math}
I_{t+1} = I_t + \beta S_t I_t - \gamma I_t
\end{math}

A makes contact sufficient to transmit infection with $r_{N}$ others per unit time, where N is the total population. Note that the probability that a random contact by an infective with a susceptible, who can then transmit infection, is S/N, therefore the number of new infections in unit time is ($r_{N}$)(S/N)I = rSI. One can also interpret $r_{P}$ in a similar way. Here r and $r_{P}$ are related to the infection rate of disease A and B respectively, while a, $a_{P}$ and b are the removal rate of individuals in class I, $I_{P}$ and E respectively. The last two equations  follow  from  the  assumption  that  the  population number equals 200. This amounts to only consider a fraction of the initial population in the Susceptibles class S (and to set $I_{P}$(0) = 0 because the spread has already  taken  place);  equivalently  this  can  be  implemented by putting at the initial time some of the population in the Removed class R (i.e., R(0) $>$ 0).


\section{Misc paragraph 4}

The total
population was taken to be S(0) = 23.67 millions; the other parameters (including the initial conditions) were optimised to ensure a good agreement with the data and were obtained to be E(0) = 2; $I_{P}$(0) = 137638; I(0) = 32;
R(0) = 0; $R_{P}$(0) = 0; r = 1.62 × 10-8 ; $r_{P}$ = 3.87 × 10-8; a = 0.120, $a_{P}$ = 0.272; b = 7.644. One approach is to use the values of a and b for all the different scenarios before the
parameters r, $r_{P}$ and a P are chosen to fit the data.


The parameters a and $a_{P}$ describe the removal from the classes I and $I_{P}$ to the classes R and $R_{P}$ respectively. Since the removed classes R are considered to contain the individuals with infections the parameters a and $a_{P}$ characterize the identification rate of potential cases; it is then rather related to the health policy than to the disease itself (e.g. a is not the mortality rate!). Since we don't know these
things, we can extract it from known doubling times $T_{d}$. The AHA says to expect a doubling time
$T_{d}$ of 7-10 days.


%\nocite{oreg,schn,pond,smith,marg,hunn,advi,koha,mouse}

%%%%%%%%%%%%%%%%%%%%%%%%%%%%%%%%%%%%%%%%%%%%%%
%%                                          %%
%% Backmatter begins here                   %%
%%                                          %%
%%%%%%%%%%%%%%%%%%%%%%%%%%%%%%%%%%%%%%%%%%%%%%

\begin{backmatter}

References

Ng, Tuen Wai, Gabriel Turinici, and Antoine Danchin. "A double epidemic model for the SARS propagation." BMC Infectious Diseases 3.1 (2003): 1-16.


\end{backmatter}
\end{document}
