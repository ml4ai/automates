\documentclass{article}

\usepackage[utf8]{inputenc}
\usepackage{natbib}
\usepackage{amssymb,amsmath}

\begin{document}

\title{Simple Epidemiological Models.}

\maketitle

\section{SIR model}

The susceptible-infected-recovered (SIR) model in a closed population was
proposed by~\cite{kermack1927contribution} as a special case of a more general
model, and forms the framework of many compartmental models.  Susceptible
individuals, $S$, are infected by infected individuals, $I$, at a per-capita
rate $\beta I$, and infected individuals recover at a per-capita rate $\gamma$
to become recovered individuals, $R$.

\begin{align}
\frac{dS(t)}{dt} &= -\beta S(t) I(t)\\
\frac{dI(t)}{dt} &= \beta S(t) I(t)- \gamma I(t)\\
\frac{dR(t)}{dt} &= \gamma I(t)
\end{align}

\section{SEIR model}

The susceptible-exposed-infected-recovered (SEIR) model extends the SIR model
to include an exposed but non-infectious class. The implementation in this
section considers proportions of susceptibles, exposed, infectious individuals
in an open population, with no additional mortality associated with infection
(such that the population size remains constant and $R$ is not modelled
explicitly).

\begin{align}
\frac{dS(t)}{dt} &= \mu-\beta S(t) I(t) - \mu S(t)\\
\frac{dE(t)}{dt} &= \beta S(t) I(t)- (\sigma + \mu) E(t)\\
\frac{dI(t)}{dt} &= \sigma E(t)- (\gamma + \mu) I(t)\\
\frac{dR(t)}{dt} &= \gamma I(t) = \mu R
\end{align}

\bibliographystyle{plain}
\bibliography{main}

\end{document}
